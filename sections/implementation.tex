\chapter{Implementation} % 
\label{chpt:implementation}


\section{Programming Paradigm} % 
\label{sec:programming-paradigm}

F\# is a functional-first programming language. This is great as functional
programming can be very succinct. But functional programming is not always the
fastest approach. For this project, the local algorithm from \cite{liu1998a} is
implemented as a \texttt{Solver}-object with an iterative implementation of the
algorithm. 

\cite{liu1998a} uses arrays in the implementation of the local algorithm. Arrays
are not very functional as they are mutable. To make the code functional, the
arrays could easily be replaced by a combination of linked lists and maps. This
will result in 1) the code being functional but also 2) the code being much
slower and taking up more memory. When interested in speed of the algorithm,
these trade-offs are severe. 

The assignments \texttt{A} and \texttt{D} in the paper are implemented as arrays
in their \texttt{Algorithm Local2}. A more apparent solution would be to
implement them as the F\# datatype \texttt{Map}. This makes sense as they more
resemble the mathematical understanding of the assignments. But Maps in F\# are
implemented as binary trees making both look-up and insertion $O(\log n)$
operations (where $n$ is the height of the tree). Insertions and look-up are
used often. 

% TODO or is $n$ the number of nodes?

The state space for this project can easily explode: Arrays are much more
compact than Maps. Maps can be more flexible but not even this is needed: We
know the entire state space before running the algorithm. 

Thus, the only not to use arrays is that they are not functional. But given that
F\# allows for iterative programming, implementing this algorithm iteratively
makes the most sense. This will make this project not purely functional, but
that is a worthy trade-off. 

\section{On-the-fly Solver} 

An assumption used for the on-the-fly solver is that the simulation relation
satisfies the following predicate: 
\begin{align}
  i_1 \neq i_2 \implies (s_1, i_1) \not \preceq (s_2, i_2)
\end{align}


\section{Domain-Specific Language (DSL)} % \label{sec:domain-specific-language}

For easier creation of railway systems a DSL has been implemented. This DSL is
inspired by the DSL described in \cite{kasting2016b}. Contrary to the approach
cited, the F\# library FsLexYacc is used to write the lexer and parser for this
DSL. 

This has been a great exercise in writing a lexer/parser. One problem with
FsLexYacc, though, is error handling: While it is very easy to spot wrong tokens
and characters in the lexer, generating a useful error message for the parser is
difficult. If this project is to expand into something bigge, this would
definitely be an area to improve. 


\section{Unit Tests} % \label{sec:unit-tests}

Some key elements of the code has been tested using unit tests. For this, the
\texttt{xUnit} library has been used.  

