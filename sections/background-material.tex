\chapter{Background Material} % 
\label{chpt:background-material}

\section{Dependency Graphs} % 
\label{sec:dependency-graphs}

A dependency graph consists of vertices and so-called hyper edges connecting the
vertices. A hyper edge points from a single vertex to zero or more vertices.
Thus, a hyper edge can be expressed as a tuple $(v, S)$ where $v$ is the vertex
and $S$ the set of vertices that this exact hyper edge points to. This implies
that it is possible that $S = \emptyset$ which is indeed the case. This is not
the same as a vertex not having a hyper edge: A vertex can have no hyper edges;
a vertex with a hyper edge pointing to the empty has at least one hyper edge.
Note that a vertex $v$ can have multiple hyper edges: $(v, S_1),\ (v, S_2),\
\ldots\ (v,S_n)$. A simple directed graph can be characterized as a dependency
graph with only singleton hyper edges. 


\section{Railway Networks} % 
\label{sec:railway-networks}

The most fundamental part of railway network is the tracks. 


